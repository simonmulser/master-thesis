The expected outcome of this thesis is a more accurate simulation of different selfish mining strategies and therefore a better understanding of the potential real world implications of such attacks.
The selfish mining strategies used in the thesis include:
\begin{itemize}
\item selfish mining \cite{eyal2014majority}
\item lead stubborn mining \cite{nayak2016stubborn}
\item trail stubborn mining \cite{nayak2016stubborn}
\item equal-fork stubborn mining \cite{nayak2016stubborn}
\end{itemize}

For the simulation, these strategies are combined with different distributions of computation power in the underlying peer-to-peer network.
The result of the simulations should show which strategy is the best strategy for a certain distribution of mining power and if the selfish mining increases the relative gain of the selfish miner compared to the normal, honest mining.
The simulation results should emphasise the recent work in the area of selfish mining and show that the current implementation of Bitcoin protocol is vulnerable against different selfish mining strategies.

The desired outcome of the thesis is supported with the following two research questions:

\begin{itemize}
	\item \textbf{RQ1:} Do the simulations of selfish mining with the proposed software solutions show an increase of the relative gain for the selfish miner compared to the normal, honest mining behaviour?

	\item \textbf{RQ2:} How does the obtained results of the simulation match the outcome of previous research in the area of selfish mining?
\end{itemize}


An additional outcome of the thesis is the simulation software.
The software should allow an accurate and deterministic simulation of the blockchain by using directly the reference implementation and a realistic network topology.
Hence, the simulation software could not only be used to simulate selfish mining attacks but could for example also be used to simulate other attacks or new protocol versions of Bitcoin. 
Since many other cryptocurrencies are derived from Bitcoin, they simulation software could be used also to simulate their behaviour and properties.
