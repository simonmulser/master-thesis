The expected outcome of this thesis is a more accurate simulation of different selfish mining strategies and therefore a better understanding of the potential real world implications of such attacks.
The selfish mining strategies include:
\begin{itemize}
\item selfish mining \cite{eyal2014majority}
\item lead stubborn mining \cite{nayak2016stubborn}
\item trail stubborn mining \cite{nayak2016stubborn}
\item equal-fork stubborn mining \cite{nayak2016stubborn}
\end{itemize}

For the simulation, these strategies are combined with different distributions of computation power and different network topologies.
The result of the simulations should show which strategy is the best strategy for a certain combination of a network topology and distribution of mining power.
Thereby, also the influence of the network topology is studied in more detail compared to previous research. The simulation results should emphasise the recent work in the area of selfish mining and show that the current implementation of Bitcoin protocol is vulnerable against different selfish mining strategies.

An additional outcome of the thesis is the simulation software.
The software should allow an accurate and deterministic simulation of the blockchain by using directly the reference implementation and a realistic network topology.
Hence, the simulation software could not only be used to simulate selfish mining attacks but could for example also be used to simulate other attacks or new protocol versions of Bitcoin. 
Since many other cryptocurrencies are derived from Bitcoin, they simulation software could be used also to simulate their behaviour and properties.
