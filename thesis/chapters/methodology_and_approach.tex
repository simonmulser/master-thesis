First, the different strategies selfish, lead stubborn, trail stubborn and equal-fork stubborn mining from \cite{nayak2016stubborn} and \cite{eyal2014majority} need to be implemented.
This is achieved by implementing a proxy which eclipses a normal Bitcoin client from the other nodes in the network.
Now, if a block is found the proxy decides, depending on his selfish mining strategy, if a block should be transmitted from the eclipsed node to the rest of the network or vice versa.
The proxy design pattern makes it possible to implement the selfish mining strategies without altering the reference implementation of Bitcoin and is therefore preferred over an implementation directly in the Bitcoin client.

In the next step, a simulation program is implemented.
To be able to control when a certain node finds a block, all Bitcoin nodes should be executed in \textit{regtest} mode.
In this test mode, the real PoW-algorithm is disabled and every node accepts a command which lets the node create immediately a new block.
With this functionality, it is possible to define a block discovery series which basically reflects the computation power of each node.
The more blocks are found by a node the more simulated computation power the node has.
Additionally to the block generation, the simulation program should also control the network topology and hence the connectivity of each node.
For the simulation run, it is important that the connectivity of the nodes stays the same to make the results better comparable.
This should be achieved by setting the connections from the nodes by the simulation program itself which is in contrast to normal behaviour.
Normally Bitcoin nodes share their connections with other nodes over the Bitcoin protocol and try to improve the connectivity over time.

After the implementation of the selfish mining strategies and the simulation program, the mining strategies are simulated.
Different network topologies and distributions of computation power are used to compare the relative gain of the selfish mining strategies over the normal, honest mining.
