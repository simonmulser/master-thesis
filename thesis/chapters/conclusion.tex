In this thesis, the selfish mining attack was simulated by using a novel simulation framework based on \textit{Docker}.
Despite previous simulation approaches, the framework allows to naturally include the network latency in the simulation and to directly reuse the reference implementation of Bitcoin.
The selfish mining itself is implemented in a separate node.
This node eclipses a normal peer of the network and conducts selfish mining by withholding blocks created by the eclipsed peer.

The results of the simulation showed that the attack increases the relative gain of the selfish miner and hence, emphasises the relevance of the selfish mining attack.
As the most promising selfish mining strategies, the normal selfish mining and equal-fork-stubbornness were identified.
In the scenario where the selfish miner has 45\% of the mining power, the two strategies were able to increase the gain of the selfish miner by 4.1\% and 5\% respectively.
