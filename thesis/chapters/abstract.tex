The cryptocurrency Bitcoin was started in 2008 with the creation of the first block, the genesis block.
Since then, the computing power of the network, which secures the blockchain of the digital currency, has multiplied considerably.
Today, around twenty professional miners are constantly extending the blockchain, always building on the latest block known to them.
The miners are incentivised to do so, as they create with each found block new Bitcoins for themselves.
In 2014, Eyal and Gün Sirer showed for the first time that apart from this desired, honest behaviour, there are deviating mining methods that increase the relative gain of a miner compared to the rest of the network.\\
\todo[inline]{imho relative gain can be understood by saying "compared to the rest of the network"}
This so-called selfish mining attack and all its modifications are examined in this thesis and compared with previous research results.
In contrast to previous research, a novel, deterministic simulation method is used, which captures the underlying peer-to-peer network and all properties of the Bitcoin protocol more accurately.
This is made possible by virtualising the entire network using the \textit{Docker} software.
Hence, the network layer of the whole system is replicated in a realistic and natural way.
Furthermore, the Bitcoin protocol can be simulated unbiased, since the actual reference implementation is executed directly in the nodes of the network.\\
The results of the simulations show the most efficient selfish mining strategy for a specific computing power of the attacker and emphasise the current state of research as well as the relevance of the selfish mining attack.
Furthermore, out of the used simulation method originated a the simulation framework, which facilitates the exploration of different attack vectors and the testing of improvements of different blockchain protocols.

\bigskip
\noindent \textbf{Keywords}\\
selfish mining, selfish mining attack, Bitcoin, blockchain, simulation, simulation method, simulation framework, network latency, reference implementation, Docker