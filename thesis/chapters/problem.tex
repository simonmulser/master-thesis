The cryptocurrency Bitcoin started back in the year 2008 with the release of the Bitcoin white paper \cite{nakamoto2008bitcoin}.
As of today, the cryptocurrency has reached a market capitalization of over 20 billion dollars \cite{marketcap2017}.
Internally the Bitcoin cryptocurrency records all transactions in a public ledger called \emph{blockchain}.
The blockchain is basically an immutable linked list of blocks where a block contains multiple transactions of the cryptocurrency.
In Bitcoin, each block needs to contain a so-called proof of work (PoW) which is the solution to a costly and time-consuming cryptographic puzzle.
Miners connected in a peer-to-peer network compete with their computation power to find solutions to the puzzle and hence to find the next block for the blockchain.
Finding a block allows the miners to add a transaction to the block and gives them right to newly create a certain amount of bitcoins.
Additionally, the grouping of the transactions in blocks creates a total order and hence makes it possible to prevent double-spending.
After a block is found by a miner, all other miners should adopt to this new tip of the chain and try to find a new block on top.
This mining process is considered as incentive compatible as long as no single miner has more than 50\% of the total computation power.

\cite{eyal2014majority} showed that also miners under 50\% have an incentive to not follow the protocol as described depending on their connectivity and share of computation power in the peer-to-peer network.
By implementing a so-called selfish mining strategy a miner can obtain relatively more revenue than his actual proportion of computational power in the network.
In general, the miner simply does not share found blocks with the others and secretly mines on his own chain.
If his chain is longer then the public chain, he is able to overwrite all blocks found by the honest miners.
If the two chains have the same length the private miner also publishes his block and causes a block race.
Now the network is split into two parts where one part is mining on the public tip and the other part is mining on the now public-private tip.
In general, the selfish miner achieves that the other miners are wasting their computational power on blocks which will not end up in the longest chain.

Further research \cite{nayak2016stubborn,sapirshtein2016optimal, gervais2015tampering, gervais2016security, bahack2013theoretical} explored different modifications of the original selfish mining algorithm by \cite{eyal2014majority} and found slightly modifications of the algorithm which perform better under certain circumstances.
For example, it could make sense for the selfish miner to even trail behind the public chain.

To prove the existence and attributes of selfish mining different approaches were applied.
The researchers used simple probabilistic arguments \cite{eyal2014majority, bahack2013theoretical}, numeric simulation of paths with state machines \cite{gervais2015tampering, nayak2016stubborn}, advanced Markov Decision Processes (MDP) \cite{sapirshtein2016optimal, gervais2016security} or gave results of closed-source simulations \cite{eyal2014majority, sapirshtein2016optimal}.
Unfortunately, we cannot discuss the closed source simulations in detail.
All other above-mentioned methodologies have the following drawbacks:
\begin{itemize}
\item Abstraction of the Bitcoin source code which normally runs on a single node.
Since there is no official specification of the Bitcoin protocol it is hard to capture all details.
Furthermore, it is hard to keep the simulation software up-to-date because of the ongoing development of the protocol.
\item Abstraction of the whole network layer of the peer-to-peer network.
The available simulations abstract the network topology by either defining a single connectivity parameter \cite{eyal2014majority, bahack2013theoretical, nayak2016stubborn, sapirshtein2016optimal, gervais2015tampering} or by using the block stale rate as input for the MDP \cite{gervais2016security}.
Hence the highly abstract the presence of network delays and natural forks of the chain.
\end{itemize}

In this thesis, we propose a new simulation approach to more accurately capture the details of the Blockchain protocol under simulation, while allowing for a high degree of determinism.
With our simulation, it would be possible to model the selfish mining attack with different network topologies and to use the Bitcoin source code directly in the simulation.