Bitcoin and cryptocurrency technologies in general are a relatively new topic.
Hence there are no concrete subjects or modules teaching this technology in the current curriculum.
But under the hood Bitcoin technically is just a composition of different technologies which can be related to modules of the curriculum.
First of all, Bitcoin is a software acting as a distributed system and can, therefore, be linked to the modules \textit{Software Engineering} and \textit{Distributed Systems}.
Furthermore, Bitcoin uses cryptography to secure the system, which can be linked to the module \textit{Advanced Security}.
Hash functions are the main component of the PoW-algorithm in the mining process which helps to prevent double spending.
Furthermore, digital signatures based on cryptography are used to secure the bitcoins held by the different users of the system.

In the thesis, the implementation of a proxy enabling selfish mining strategies and a simulation program are carried out.
Since both of them are an implementation effort they can be directly linked to the module \textit{Software Engineering}.
Furthermore, both software programs are related to the module \textit{Distributed Systems}.
The proxy needs to be suited between multiple nodes in a Bitcoin network which is a distributed system and the simulation program needs to start-up, manage and tear-down exactly this distributed system.
