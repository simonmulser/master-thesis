\todo[inline]{the positioning of code listings does not work automagically. i will position them when the rest of the thesis is fine.}

The cryptocurrency Bitcoin started back in the year 2008 with the release of the Bitcoin white paper \cite{nakamoto2008bitcoin}.
As of today, the cryptocurrency has reached a market capitalization of over 20 billion dollars \cite{marketcap2017}.
Internally the Bitcoin cryptocurrency records all transactions in a public ledger called \emph{blockchain}.
The blockchain is basically an immutable linked list of blocks where a block contains multiple transactions of the cryptocurrency.
In Bitcoin, each block needs to contain a so-called proof-of-work (PoW) which is the solution to a costly and time-consuming cryptographic puzzle.
Miners connected in a peer-to-peer network compete with their computation power to find solutions to the puzzle and hence to find the next block for the blockchain.
Finding a block allows the miners to add a transaction to the block which gives them a certain amount of Bitcoins.
Additionally, the grouping of the transactions in blocks creates a total order and hence makes it possible to prevent double-spending.
After a block is found by a miner, all other miners should adopt this new tip of the chain and try to find a new block on top.
This mining process is considered as incentive compatible as long as no single miner has more than 50\% of the total computation power.

\cite{eyal2014majority} showed that miners under 50\% have an incentive to not follow the protocol as described depending on their connectivity and share of computation power in the peer-to-peer network.
By conducting a so-called selfish mining strategy a miner can obtain relatively more revenue than its actual proportion of computational power in the network.
In general, the miner simply does not share found blocks with the others and secretly mines on its own chain.
If its chain is longer then the public chain, it is able to overwrite all blocks found by the honest miners.
If the two chains have the same length the private miner also publishes its block and causes a block race.
Now the network is split into two parts where one part is mining on the public tip and the other part is mining on the now public-private tip.
In general, the selfish miner achieves that the other miners are wasting their computational power on blocks which will not end up in the longest chain and thus is able to increase its relative share of mining rewards.

Even though the selfish miner can increase its relative gain compared to the other miners, its total gain is lower than it would behave honestly.
This is possible because the execution of the selfish mining algorithm by the selfish miner decreases the number of blocks which end up in the longest chain.
During selfish mining more block, of the selfish miner as well as the honest network, are not included in the longest chain, the so-called stale block rate increases.
Hence, less mining rewards are distributed between the nodes in the network which also affects the miner conducting selfish mining.
The fact that less mining rewards are distributed prevails the relative gain obtained by the selfish miner and therefore the selfish miner earns less than it would earn behaving honestly \cite{nayak2016stubborn, sapirshtein2016optimal}.
Since there end up fewer blocks in the longest chain the Bitcoin protocol would simplify the cryptographic puzzle during the next difficulty adjustment.
From then on, more blocks are found and it is likely that the selfish miner would also increase its total gain.
The scenario where the difficulty is adjusted was not yet investigated by researchers or observed on a blockchain network like Bitcoin and is also not a focus of this thesis.

Despite the normal selfish mining further research \cite{nayak2016stubborn,sapirshtein2016optimal, gervais2015tampering, gervais2016security, bahack2013theoretical} showed different modifications of the algorithm which perform slightly better under certain circumstances.
These modifications include for example the idea that the selfish miner does not adopt immediately to new blocks but trails behind the public chain and tries to catch up with its secretly mined, private chain.
Furthermore, the selfish mining attack can be combined with different other attacks such as double spending increasing the gain of the misbehaving miner \cite{gervais2016security, sapirshtein2016optimal, nayak2016stubborn, gervais2015tampering}.

To prove the existence and attributes of selfish mining different approaches were applied.
The researchers used simple probabilistic arguments \cite{eyal2014majority, bahack2013theoretical}, numeric simulation of paths with state machines \cite{gervais2015tampering, nayak2016stubborn}, advanced Markov Decision Processes (MDP) \cite{sapirshtein2016optimal, gervais2016security} or gave results of closed-source simulations \cite{eyal2014majority, sapirshtein2016optimal}.
Unfortunately, we cannot discuss the closed source simulations in detail.
All other above-mentioned methodologies have the following drawbacks:
\begin{itemize}
\item Abstraction of the Bitcoin reference implementation which runs inside a node of the peer-to-peer network.
Since there is no official specification of the Bitcoin protocol it is hard to capture all details.
Furthermore, it is hard to keep the simulation framework up-to-date because of the ongoing development of the protocol.
\item Abstraction of the whole network layer of the peer-to-peer network.
The available simulations abstract the network topology by either defining a single connectivity parameter \cite{eyal2014majority, bahack2013theoretical, nayak2016stubborn, sapirshtein2016optimal, gervais2015tampering} or by additionally using the stale block rate as input for the MDP \cite{gervais2016security}.
Hence they highly abstract the presence of network delays and natural forks of the chain.
\end{itemize}

In this thesis, we propose a new simulation approach which tackles this two drawbacks and hence captures more accurately the details of the Bitcoin reference implementation and the whole network layer while allowing for a high degree of determinism.
The introduced simulation approach is then used to examine the following selfish mining strategies:

\begin{itemize}
\item selfish mining \cite{eyal2014majority}
\item lead stubborn mining \cite{nayak2016stubborn}
\item trail stubborn mining \cite{nayak2016stubborn}
\item equal-fork stubborn mining \cite{nayak2016stubborn}
\end{itemize}

These strategies are executed by a selfish mining node in a defined, realistic simulation scenario with a certain amount of participating nodes.
To gain a holistic insight different distributions of computation power between the selfish miner and the public network are used.

The result of the executed simulations shows which strategy is the best strategy for a certain distribution of mining power in the simulation scenario.
Furthermore, the relative and total gain of the selfish miner is observed and compared with previous research.
The evaluation of the simulation results is sustained with following two research questions:

\begin{itemize}
	\item \textbf{RQ1:} Do the simulations of selfish mining with the proposed software solutions show an increase of the total and relative gain for the selfish miner compared to the normal, honest mining behaviour?

	\item \textbf{RQ2:} How do the obtained results of the simulation match the outcome of previous research in the area of selfish mining?
\end{itemize}


An additional outcome of the thesis is a general simulation framework.
The implemented framework allows an accurate and near-deterministic simulation of the blockchain by using directly the Bitcoin reference implementation and a realistic network topology.
%Hence, the simulation framework could not only be used to simulate selfish mining attacks but could, for example, be used to simulate other attacks or new protocol versions of Bitcoin. 
Hence, the simulation framework could not only be used to simulate selfish mining attacks but could also be used to simulate other attacks as well, on different protocol versions of Bitcoin. 
Since many other cryptocurrencies are derived from Bitcoin, the simulation framework could also be utilized to simulate their behaviour and properties.

\section{Structure of this thesis}

\todo[inline]{reorder to reflect ToC and added paragraph for evaluation of the framework}

First, a simulation framework is implemented.
To be able to control when a certain node finds a block, all Bitcoin nodes are executed in \textit{regtest} mode.
In this test mode, the real PoW-algorithm is disabled and every node accepts a command which lets the node create immediately a new block.
With this functionality, it is possible to define a block discovery series which basically reflects the computation power of each node.
The more blocks are found by a node the more simulated computation power the node has.
Additionally to the block generation, the simulation framework also controls the network topology and hence the connectivity of each node.
For the simulation run, it is important that the connectivity of the nodes stays the same to make the results better comparable.
This is achieved by setting the connections from the nodes by the simulation framework itself which is in contrast to the normal behaviour.
Normally Bitcoin nodes share their connections with other nodes over the Bitcoin protocol and try to improve the connectivity over time.

Subsequently, the degree of determinism of the simulation framework is evaluated.
Therefore a near-deterministic behaviour and a realistic reference scenario reflecting the Bitcoin network are defined.
The results of executing the scenario with the framework show then that the framework behaves near-deterministic as defined.

In the next step, the different strategies selfish, lead stubborn, trail stubborn and equal-fork stubborn mining from \cite{nayak2016stubborn} and \cite{eyal2014majority} are implemented.
This is achieved by implementing a proxy which eclipses a normal Bitcoin client from the other nodes in the network.
Now, if a block is found the proxy decides, depending on its selfish mining strategy, if a block should be transmitted from the eclipsed node to the rest of the network or vice versa.
The proxy design pattern makes it possible to implement the selfish mining strategies without altering the reference implementation of Bitcoin and is therefore preferred over an implementation directly in the Bitcoin client.

Lastly, the simulation framework and the selfish proxy is used to simulate the various selfish mining strategies.
Therefore the reference scenario defined for evaluating the framework is combined with different distributions of computation power between the selfish miner and the honest network.
The results of this simulations are then analysed and compared with previous research in this area.
