what are we doing now?
 - describing the simulation software
 - simulation software exposes six commands
   - nodes
   - network
   - connectivity
   - ticks
   - simulate
   - run
   - multi-run

\section{Configuration files}

nodes
 - let you create groups of node with the same configuration
 - group has type, amount, share, latency, docker image
 - groups are persisted in a csv
 
network
 - let you create a network configuration
 - takes nodes.csv and connects them based on defined connectivity
 - network configuration is persisted in a csv

ticks
 - let you create a certain amount of ticks
 - a tick contains block events
 - using a exponential distribution and depending on the computational share it is determined when a node finds a block
 - ticks are persisted in a csv

component diagram of simulation software
 - simulationfiles
 - run, multi-run
 -  prepare
 - execution
 - post processing
 
 flow chart of run command

\section{Simulation}

 - executed eg. with simulate cmd
 - in the simulation docker is used
 - each simulation has three todo{phases/steps (which word should we use?)}
 - takes configuration files as input 

\subsection{Preparation}
 - nodes are created and connected
 - create RPC-connection
 - wait until ready
\subsection{Execution}
 - execute ticks from ticks.csv line by line
 - tick duration defined
 - execute all events in tick sequentially
 - afterwards sleep until tick is over, then start over again
 - during execution CPU \& RAM usage is collected
\subsection{Post processing}
 - get chaintips of all nodes
 - calculate consensus chain
   - start at first block height
   - request from all nodes blocks
   - if every node has a block and all blocks are the same, add block to consensus chain and heigt++
   - otherwise stop
 - stop nodes
 - process log file of every node and log file of simulation software
 - parse log events:
   - BlockCreateEvent
   - BlockStatsEvent
   - UpdateTipEvent
   - PeerLogicValidationEvent
   - TxEvent
   - TickEvent
   - BlockReceivedEvent
   - BlockReconstructEvent
   - TxReceivedEvent
   - BlockExceptionEvent
   - TxExceptionEvent
   - RPCExceptionEvent
 - pre-processing csv
   - need to have one report.rmd for run and multi-runs
   - remove skip-ticks
   - files are not sorted because of multi-processed parsing; sort by timestamp 
 - report creation
   - with Rmarkdown, R ...
   - calculate propagation time of blocks
   - calculate stale blocks
   - create useful stats about CPU \& RAM usage, duration, blocks, propagation, exceptions...
   
\subsection{Multi-runs}
 - runs a simulation multiple times with the same arguments
 - merges csv of all runs by type
 - create a report showing all results of all runs