Die Cryptocurrency Bitcoin wurde im Jahre 2008 mit dem Erstellen des ersten Blocks, dem Genesis Block, gestartet.
Seitdem hat sich die Rechenleistung des Netzwerkes, welches die Blockchain der digitalen Währung absichert, erheblich vervielfacht.
Heutzutage erweitern um die zwanzig professionelle Miner laufend die Blockchain, indem sie immer auf den jeweils neuesten, ihnen bekannten Block aufbauen.
Die Miner belohnen sich dabei selbst, da sie mit jedem erstellten Block für sich neue Bitcoins schürfen.
Im Jahre 2014 zeigte Eyal und Gün Sirer erstmals dass es neben diesem gewünschtem, ehrlichen Verhalten abweichende Miningmethoden gibt, welche den relativen Ertrag eines Miners gegenüber seiner Kontrahenten erhöht.\\
Dieser sogenannte Selfish-Mining-Angriff und all seine	 Modifikationen werden in der vorliegenden Diplomarbeit untersucht und mit bisherigen Forschungsresultaten verglichen.
Im Gegensatz zu den bisherigen Forschungsarbeiten wird dafür eine neuartige, deterministische Simulationsmethode verwendet, welche das zugrunde liegende Peer-to-Peer-Netzwerk und alle Eigenschaften des Bitcoin-Protokolls akkurater abbildet.
Dies wird ermöglicht, indem das gesamte Netzwerk mittels der Software \textit{Docker} virtualisiert wird.
Dadurch wird der Netzwerklayer des Systems und dessen Latenz auf natürliche Art und Weise nachgebildet.
Weiteres kann das Bitcoin-Protokoll realitätsnah simuliert werden, da die Referenzimplementierung direkt in den Nodes des Netzwerkes ausgeführt wird.\\
%Neben der Berücksichtigung aller Details des Protokolls erspart man sich dabei zusätzlich eine fehleranfällige und zeitaufwendige Adaptierung oder Neuimplementierung des Protokolls.\\
Die Resultate der Simulationen zeigen die jeweils effizienteste Selfish-Mining-Strategie für eine bestimme Rechenleistung des Angreifers und untermauern den momentanen Forschungsstand sowie die Relevanz des Selfish-Mining-Angriffs.
Weiters wurde durch die implementierte Simulationsmethode ein neues Simulationsframework entwickelt, welches das Erforschen verschiedenster Angriffsvektoren und Testen von Verbesserungen unterschiedlicher Blockchain-Protokolle erleichtert.

\bigskip
\noindent \textbf{Schlagwörter}\\
Selfish-Mining, Selfish-Mining-Angriff, Bitcoin, Blockchain, Simulation, Simulationsmethode, Simulationsframework, Netzwerklatenz, Referenzimplementierung, Docker
